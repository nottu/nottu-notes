\documentclass{article}
\usepackage[margin=0.5in]{geometry}
\usepackage{listings}
\usepackage{color}
\usepackage[utf8]{inputenc}
\usepackage[spanish]{babel}
\usepackage{multicol}
\usepackage{algorithm}
\usepackage{algpseudocode}
\usepackage{graphicx}
\usepackage{helvet}
\usepackage{amsmath}
\usepackage{amssymb}
\newcommand{\R}{\mathbb{R}}
\renewcommand{\familydefault}{\sfdefault}
\graphicspath{ {./} }

\title {Optimización Tarea 3}
  \author {Francisco Javier Peralta Ramírez}
  \date{\vspace{-2ex}}

\begin{document}
\maketitle

\begin{enumerate}
\item Escribe la expansión de Taylor de segundo orden para
$$f(\boldsymbol{x}) = 100(x_2-x_1^2)^2 + (1 - x_1)^2 \qquad \boldsymbol{x} in \mathbb{R}^2$$
\begin{enumerate}
\item La expansión de Taylor de segundo orden para variables multidimensionales se escribe como
$$f(\boldsymbol{x}) \approx f(\boldsymbol{a}) + Df(\boldsymbol{a})(\boldsymbol{x - a}) + \frac{1}{2}Hf(\boldsymbol{a})(\boldsymbol{x - a})$$
Donde $Hf(\boldsymbol{a}) = DDf(\boldsymbol{a}).$ es decir la \emph{matriz Hessiana}.
\item En el caso de nuestra función, podemos calcular la primera derivada y el Hessiano

$$
Df(\boldsymbol{x})= 
 \begin{pmatrix}
  400(x_2 - x_1^2)x_1 + 2(1 - x_1) &
  200(x_2 - x_1^2)\\
 \end{pmatrix}
$$

$$
Hf(\boldsymbol{x})= 
 \begin{pmatrix}
  400 x_2 - 600x_1^2 - 2 && 400x_1\\
  400x_1 & 200\\
 \end{pmatrix}
$$

Ahora podemos substituir en la formula anterior 

$$f(\boldsymbol{x}) \approx f(\boldsymbol{a}) + \begin{pmatrix}
  400(a_2 - a_1^2)a_1 + 2(1 - a_1) \\
  200(a_2 - a_1^2)\\
 \end{pmatrix}^T
 \begin{pmatrix}
  x_1 - a_1\\
  x_2 - a_2\\
 \end{pmatrix} + \frac{1}{2} \begin{pmatrix}
  400 x_2 - 600x_1^2 - 2 && 400x_1\\
  400x_1 & 200\\
 \end{pmatrix}\begin{pmatrix}
  x_1 - a_1\\
  x_2 - a_2\\
 \end{pmatrix}$$

\end{enumerate}
\item Supngamos que observamos $m$ valores de una función $g$ en los puntos $x_1, x_2, \cdots , x_m$ por lo que conocemos los valores de $g(x_1), g(x_2), \cdots , g(x_m)$. Queremos aproximar la función $g(\cdot) : \mathbb{R} \rightarrow \mathbb{R}$ por un polinomio
$$h_n(x) = a_0 + a_1x + a_2x^2 + \cdots + a_nx^n$$
con $n < m$

\begin{enumerate}
  \item Genera observaciones del modelo
  $$g(x) = \frac{sin(x)}{x} + \eta \qquad x \in [0.1, 10]$$
  Donde $\eta \sim \mathcal{N}(0, 0.1) 0.1=x_1 <x_2 <\cdots<x_m =10.$
  \item Aproxima $g(\cdot)$ por $h_n(x)$ para $n = 2, \cdots, 5$
  \item Grafica en la misma gráfica $h_n(x)$ para $n = 2, \cdots, 5$ y ${x_i, g(x_i)}, i = 1, 2, \cdots, m$
\end{enumerate}

\newpage
\item Dada una $f(x)$ continua en $[a, b]$ encuentra el polinomio de aproximación de grado n 
$$p_n(x) = a_0 + a_1x + a_2x^2 + \cdots + a_nx^n$$
que minimiza
$$\int\limits_a^b[f(x) - p(x)]^2 dx$$

\item Para la distribución normal $\mathcal{N}(\mu, \sigma^2)$ que tiene una función de densidad de probabilidad 

$$f(x | \mu, \sigma^2) = \frac{1}{\sqrt{2\pi\sigma^2}}exp\left(-\frac{(x-\mu)^2}{2\sigma^2}\right)$$

La función de desindad de probabilidad correspondiente para una muestra $\{x_i\}_{i=1}^n$ de n variables independientes e identicamente distribuidas es (verosimilitud):
$$\mathcal{L}(\mu, \sigma^2; x) = f(x_1, x_2, \cdots, x_n | \mu,\sigma^2) = \prod_{i=1}^{n}f(x_i| \mu,\sigma^2)$$

Con $x = x_1, x_2, \cdots, x_n$. Calcula
$$(\mu^*, \sigma^*) = arg \max_{\mu, \sigma} \mathcal{L}(\mu, \sigma^2; x)$$
\begin{enumerate}
\item Suponga que $f : \mathbb{R} \rightarrow \mathbb{R}$ es convexa y $a, b \in dom f$ con $a < b$. Muestra que 
$$ a(f(x) - f(b)) + x(f(b) - f(a)) + b(f(a) - f(x)) \geq 0 \qquad \forall x \in [a, b]$$

\item Suponga que $f : \mathbb{R}^n \rightarrow \mathbb{R}$ es convexa con $\boldsymbol{A} \in \mathbb{R}^{n\times m}$ y $\boldsymbol{b} \in \mathbb{R}^n$. Muestra que la función $g: \mathbb{R}^m \rightarrow \mathbb{R}$ definida por $g(\boldsymbol{x}) = f(\boldsymbol{Ax}+\boldsymbol{b})$, con $dom \,g = \{x |\boldsymbol{Ax}+\boldsymbol{b} \in dom \,f\}$, es convexa.
\end{enumerate}

\end{enumerate}

\end{document}