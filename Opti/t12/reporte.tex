\documentclass{article}
\usepackage[margin=0.7in]{geometry}
\usepackage{listings}
\usepackage{color}
\usepackage[utf8]{inputenc}
\usepackage[spanish]{babel}
\usepackage{multicol}
\usepackage{algorithm}
\usepackage{algpseudocode}
\usepackage{graphicx}
\usepackage{helvet}
\usepackage{amsmath}
\usepackage{amssymb}
\newcommand{\R}{\mathbb{R}}
\renewcommand{\familydefault}{\sfdefault}
\usepackage{caption}
\usepackage{subcaption}
\usepackage[table,xcdraw]{xcolor}
\usepackage{enumitem}

\decimalpoint
\graphicspath{ {./} }

\title {Optimización Tarea 12}
  \author {Francisco Javier Peralta Ramírez}

\begin{document}
\maketitle
\begin{abstract}
  En está tarea se resuelven algunos problemas de optimización con retricciones usando el método de multiplicadores penalización cuadrática y Lagrangiano aumentado. También se progrmó el método de penalización cuadrática y se probó resolviendo un problema con multiples restricciones.
\end{abstract}
\begin{multicols}{2}

\section*{Introducción}

 El método de penalización cuadrática consiste en genera una nueva función objetivo ($q$) usando la función objetivo original más un término adicional por cada restricción. Estos terminos ligados a la restricción son positivos cuando se evalúa en un punto que viola la restricción. A estos términos se les conoce como términos de penalización.

La función generada por este método es de la forma
\vspace{-2ex}
$$q(\boldsymbol{x}, \mu) = f(\boldsymbol{x}) + \frac{\mu}{2} \sum_{i=1}^{m} g_i^2(\boldsymbol{x}) + \frac{\mu}{2} \sum_{i=1}^{l} [h_i(\boldsymbol{x})^-]^2$$
\noindent donde $[v^-] = \max\{v, 0\}$

 Para la programación de esté método se creo una función que calcula el gradiente de la función $q$ donde $\frac{\partial f(\boldsymbol{x})}{\partial x_i} \approx \frac{ f(\boldsymbol{x} + h\boldsymbol{e_i}) - f(\boldsymbol{x}) }{h}$ donde $\boldsymbol{e_i}$ es el i-ésimo vector canónico.


El método de Lagrange Aumentado es una combinación del método de lagrange con el método de penalización cuadrática. Se crea una función objetivo de la forma
\vspace{-2ex}
$$ \mathcal{L}_A(\boldsymbol{x};\boldsymbol{\lambda}; \mu) = f(\boldsymbol{x}) - \sum_{i=1}^{m}\lambda_ig_i(\boldsymbol{x}) + \frac{\mu}{2} \sum_{i=1}^{m} g_i^2(\boldsymbol{x})$$

\section*{Penalización Cuadrática}

Resolvemos el problema 
\begin{align*}
\min_{\boldsymbol{x}}&\quad x_1^2 + 2 x_2^2\\
\text{ sujeto a }&\quad 1-x_1+x_2 \leq 0
\end{align*}
usando el método de penalización cuadrática calculando el límite de $\mu \rightarrow \infty $
\begin{align*}
q(\boldsymbol{x}, \mu) = x_1^2 + 2 x_2^2 + \mu[\max\{ 0, 1-x_1 + x_2 \}]^2\\
\end{align*}
Calculamos las derivadas parciales cuando no se cumple la restricción, es decir $1-x_1 + x_2 > 0$.
\begin{align*}
\frac{\partial q(\boldsymbol{x}, \mu)}{\partial x_1} &= 2x_1 - 2\mu(1-x_1+x_2)\\
\frac{\partial q(\boldsymbol{x}, \mu)}{\partial x_1} &= 4x_2 - 2\mu(1-x_1+x_2)\\
\end{align*}
Igualando ambas equaciones a $0$ obtenemos un sistema de equaciones. Con $\frac{\partial q(\boldsymbol{x}, \mu)}{\partial x_1} + \frac{\partial q(\boldsymbol{x}, \mu)}{\partial x_2}$ obtenemos que $2x_1 + 4x_2 = 0 \rightarrow x_1 = -2x_2$. Substituyendo en $\frac{\partial q(\boldsymbol{x}, \mu)}{\partial x_1}$
\begin{align*}
4x_2 - 2\mu(1-x_1+x_2) &=0\\
4x_2 - 2\mu(1+2x_2+x_2) &=0\\
4x_2 - 2\mu +6\mu x_2 &=0\\
2x_2(2+3\mu) &= -2\mu\\
x_2 &= -\frac{\mu}{2+3\mu}\\
x_1 &= \frac{2\mu}{2+3\mu}
\end{align*}
Calculamos los límites cuando $\mu \rightarrow \infty$
\begin{align*}
\lim_{\mu\rightarrow\infty} -\frac{\mu}{2+3\mu} &= -\frac{\infty}{\infty}\\
\text{L'Hopital}\rightarrow \lim_{\mu\rightarrow\infty} -\frac{1}{3} &= -\frac{1}{3}\\
\lim_{\mu\rightarrow\infty} \frac{2\mu}{2+3\mu} &= \frac{\infty}{\infty}\\
\text{L'Hopital}\rightarrow \lim_{\mu\rightarrow\infty} \frac{2}{3} &= \frac{2}{3}
\end{align*}
podemos caulcular los eigenvalores de la función $f(\boldsymbol{x}) + \mu p(\boldsymbol{x})$ y calcular su límite cuando $\mu \rightarrow \infty$. Tomando el caso cuando no se cumple la restricción:
\begin{align*}
\frac{\partial q(\boldsymbol{x}, \mu)}{\partial^2 x_1}   &= 2 + 2\mu\\
\frac{\partial q(\boldsymbol{x}, \mu)}{\partial x_1 x_2} &= -2\mu\\
\frac{\partial q(\boldsymbol{x}, \mu)}{\partial x_1 x_2} &= 4 + 2\mu\\
\end{align*}
Recordamos que para encontrar los eigenvalores basta con resolver $\det(\nabla^2q - \lambda I) = 0$
$$
\nabla^2q - \lambda I= 
 \begin{pmatrix}
  2 + 2\mu -\lambda && -2\mu\\
  -2\mu             && 4 + 2\mu - \lambda\\
 \end{pmatrix}
$$
Tomamos $H = \nabla^2q - \lambda I$
\begin{align*}
\det(H) &= (2 + 2\mu -\lambda)(4+2\mu - \lambda) - (-2\mu)^2 = 0\\
&= 8+12\mu-6\lambda-4\mu\lambda+4\mu^2+\lambda^2 - 4\mu^2 = 0\\
&= 8+12\mu + \lambda^2 - (4\mu+6)\lambda = 0
\end{align*}
Usando la formula general para encontrar $\lambda$ obtenemos $\lambda = 2\mu + 3 \pm [4\mu^2 + 1]^{1/2}$. Calculamos los limites
\begin{align*}
\lim_{\mu\rightarrow\infty} 2\mu + 3 + [4\mu^2 + 1]^{1/2} &= \infty\\
\lim_{\mu\rightarrow\infty} 2\mu + 3 - [4\mu^2 + 1]^{1/2} &= \infty - \infty\\
\text{racionalizamos}\\
\lim_{\mu\rightarrow\infty} 2\mu + 3 - [4\mu^2 + 1]^{1/2} &\frac{2\mu + 3 + [4\mu^2 + 1]^{1/2}}{2\mu + 3 + [4\mu^2 + 1]^{1/2}}\\
\lim_{\mu\rightarrow\infty} \frac{(2\mu + 3)^2 - [4\mu^2 + 1]}{2\mu + 3 + [4\mu^2 + 1]^{1/2}}\\
\lim_{\mu\rightarrow\infty} \frac{4\mu^2 + 12\mu +9 - 4\mu^2 - 1}{2\mu + 3 + [4\mu^2 + 1]^{1/2}}\\
\lim_{\mu\rightarrow\infty} \frac{12\mu +8}{2\mu + 3 + [4\mu^2 + 1]^{1/2}}\\
\text{ multiplicamos por} \frac{1/\mu}{1/\mu}\\
\lim_{\mu\rightarrow\infty} \frac{12 +8/\mu}{2 + 3/\mu + [4 + 1/\mu^2]^{1/2}}\\
&= \frac{12}{2+\sqrt{4}} = \frac{12}{4}= 3
\end{align*}
Con esto podemos calcular el condicionamiento del Hessiano $K_2 = \infty/3$ por lo cual podemos concluir que se mal condiciona cuando $\mu \rightarrow \infty$

\section*{Lagrangiano Aumentado}

Tenemos el problema

\begin{align*}
\min_{\boldsymbol{x}}&\quad 3x_1^2 + x_2^2\\
\text{ sujeto a }&\quad x_1+2x_2 - 1 \leq 0
\end{align*}

La función
$$L_A(\boldsymbol{x}, \lambda, \mu) = 3x_1^2 + x_2^2 - \lambda(x_1+2x_2 - 1) + \frac{\mu}{2}(x_1+2x_2 - 1)^2$$

$x_1 = \frac{mu + \lambda}{6+13\mu}$, $x_2 = \frac{6mu + 6\lambda}{6+13\mu}$ Cuando $\mu$ tiende a infinito

$x_1 = \frac{1}{13}$ y $x_2 = \frac{6}{13}$. Y podemos obtener el valor de $\lambda = 6/13$

\section*{Penalización Cuadrática Programado}

Se programó el método de penalización cuadrática, y se usó el método de máximo descenso con paso backtracking para optimizar la función $q$ ,de esta forma no necesitamos el hessiano y podemos optimizar sólo con nuestra aproximación del gradiente.

Obtenemos el resultado en cuatro iteraciones:

x : (3.04411 4.02629 )   f:17.5802  h\_1 : -11.1849  h\_2 : -2.01782  h\_3 : 0.0704082 h\_4 : -3.33022

\section*{Conclución}
El método de penalización cuadrática nos permite resolver problemas con restricciones usando una nueva función objetivo, la cual podemos resolver con métodos que ya teníamos. Esto facilita la programación del método y sólo requiere hacer pequeños ajustes para que funcione con nuestro código previo.
\newpage

\end{multicols}
\end{document}